\documentclass[border = 1cm, preview, varwidth=\maxdimen]{standalone}

\usepackage{xeCJK}

% mathematics
\usepackage{amsmath}

% tikz
\usepackage{tikz}
\usepackage{ifthen}
\usetikzlibrary{arrows}
\usetikzlibrary{automata}
\usetikzlibrary{positioning}
\tikzset{
  ->,
  >=stealth',
  node distance = 1.5cm,
}

\begin{document}
\begin{tikzpicture}
  % nodes
  \node [state, initial] (s0) {$s_0$};
  \foreach \curr in {1, ..., 9} {
    \pgfmathtruncatemacro \prev { \curr - 1 };
    \ifthenelse {\curr = 5}
    {\node [state, below of = s\prev] (s\curr) {$s_{\curr}$};}
    { \ifthenelse {\curr < 5}
        {\node [state, right of = s\prev] (s\curr) {$s_{\curr}$};}
        {\node [state, left of = s\prev] (s\curr) {$s_{\curr}$};}
    }
  }
  \node [state, accepting, left of = s9] (s10) {$s_{10}$};

  % paths
  \foreach \char [count=\k] in {
      辅音, \textbf a, 辅音, \textbf e, 辅音, \textbf i,
      辅音, \textbf o, 辅音, \textbf u, 辅音} {
    \pgfmathtruncatemacro \curr { \k - 1 };
    \ifthenelse {\curr < 5}
      {\draw (s\curr) edge [loop above] node {\char} (s\curr);}
      {\draw (s\curr) edge [loop below] node {\char} (s\curr);}
  }
  \foreach \curr in {2, 4, ..., 10} {
    \pgfmathtruncatemacro \prev { \curr - 1 };
    \draw (s\prev) edge [above] node {辅音} (s\curr);
  }
  \draw (s0) edge [above] node {\textbf a} (s1);
  \draw (s1) edge [below, bend right] node {\textbf e} (s3);
  \draw (s2) edge [above] node {\textbf e} (s3);
  \draw (s3) edge [above] node {\textbf i} (s5);
  \draw (s4) edge [left] node {\textbf i} (s5);
  \draw (s5) edge [above, bend right] node {\textbf o} (s7);
  \draw (s6) edge [above] node {\textbf o} (s7);
  \draw (s7) edge [above, bend right] node {\textbf u} (s9);
  \draw (s8) edge [above] node {\textbf u} (s9);
\end{tikzpicture}
\end{document}
